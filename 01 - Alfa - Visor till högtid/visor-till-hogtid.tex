\documentclass[a6paper, 10pt, twoside]{article}
%\usepackage[T1]{fontenc}
\usepackage[british]{babel}
\usepackage[utf8]{inputenc}
\usepackage{float, graphicx,amsmath,amsfonts,cite,enumerate}
\usepackage[final]{pdfpages}
\usepackage{wrapfig}
\usepackage[margin=0.3in]{geometry}
\usepackage{sidspaltHack}
\usepackage{digital}

\setlength{\oddsidemargin}{-0.37in}
\setlength{\evensidemargin}{-0.47in}
\setlength{\textwidth}{215pt}

\pagestyle{empty}

\begin{document}
\noindent
\chaptertitlenobr{A$\alpha$}{Visor till högtid}
\small
\begin{center}
    \songtitle{$\alpha1$}{Du gamla du fria}
    \songsubtitle{(Till svenska fosterjorden)}
    \sheetmusicnotice{Noter till blandad kör finns i notkapitlet.}
\end{center}
\begin{lyrics}
Du gamla du fria, du fjällhöga nord. \\
Du tysta, du glädjerika sköna. \\
Jag hälsar dig vänaste land uppå jord.\\
$\|$: Din sol, din himmel, dina ängder gröna. :$\|$\\
\newline
Du tronar på minnen från fornstora dar\\
då ärat ditt namn flög över jorden.\\
Jag vet att du är och du blir vad du var.\\
$\|$: Ja, jag vill leva jag vill dö i Norden. :$\|$\\
\digitalonly{
Jag städs vill dig tjäna mitt älskade land,\\
Dig trohet till döden vill jag svära.\\
Din rätt skall jag värna med håg och med hand,\\
$\|$: Din fana, högt din bragderika bära. :$\|$ \\
\newline
Med Gud skall jag kämpa, för hem och för härd,\\
för Sverige, den kära fosterjorden.\\
Jag byter Dig ej, mot allt i en värld\\
$\|$: Nej, jag vill leva jag vill dö i Norden! :$\|$\\
}
\end{lyrics}
\auth{Text: Rikard Dybeck}

\nysida{1}{2}
\begin{center}
    \songtitle{$\alpha2$}{Kungssången (Hymn)}
    \sheetmusicnoticefootnote{Noter till manskör finns i notkapitlet}
\end{center}
\begin{lyrics}
Ur svenska hjärtans djup en gång.\\
En samfälld och en enkel sång\\
som går till Kungen fram.\\
Var honom trofast och hans ätt.\\
Gör kronan på hans hjässa lätt.\\
Och all din tro till honom sätt\\
du folk av frejdad stam.\\
\newline
Du himlens herre med oss var,\\
som förr du med oss varit har\\
och liva på vår strand.\\
Det gamla lynnets art igen\\
hos sveakungen och hans män,\\
Och låt den ande vila än\\
utöver nordanland.\\
\end{lyrics}
\auth{Text: C. V. A. Strandberg\\Musik: Otto Lindblad.}

\nysida{1}{3}
\begin{center}
    \songtitle{$\alpha3$}{Sveriges flagga}
    \sheetmusicnotice{Noter finns i notkapitlet.}
\end{center}
\begin{lyrics}
Flamma stolt mot dunkla skyar\\
lik en glimt av sommarns sol.\\
Över Sveriges skogar, berg och byar,\\
över vatten och viol.\\
Du, som sjunger, när du bredes,\\
som vår gamla lyckas tolk:\\
"Solen lyser! Solen lyser!\\
Ingen vredes åska slog vårt tappra folk!"
\vspace{5pt}\\
Flamma högt, vår kärleks tecken,\\
värm oss, när det blåser kallt!\\
Stråla ur de blåa vecken,\\
kärlek, mera stark än allt!\\
Sveriges flagga, Sveriges ära,\\
fornklenod och framtids tolk:\\
Gud är med oss, Gud är med oss\\
Hen skall bära starkt vårt fria svenska folk. 
\end{lyrics}
\auth{Text: KG Ossiannilsson\\Musik: Hugo Alfvén}
\vspace{20pt}
\begin{figure}[!h]
\centering
\includegraphics[width=0.6\textwidth]{sveriges_flagga.png}
\end{figure}

\nysida{1}{4}
\begin{center}
    \songtitle{$\alpha4$}{Internationalen}
\end{center}
\begin{lyrics}
Upp trälar uti alla stater, \\
som hungern bojor lagt uppå. \\
Det dånar uti maktens krater, \\
snart skall frihetens timma slå. \\
Störtas skall det gamla uti gruset. \\
Slav, stig upp för att slå dig fri! \\
Från mörkret stiga vi mot ljuset, \\
från intet allt vi vilja bli. \\
\newline
$\|$: Upp till kamp emot kvalen \\
sista striden det är. \\
ty Internationalen \\
åt alla lycka bär. :$\|$\\
\newline
I höjden räddarn vi ej hälsa, \\
ej gudar, furstar stå oss bi. \\
Nej, själva vilja vi oss frälsa, \\
och samfälld skall vår räddning bli. \\
För att kräva ut det stulna, bröder, \\
och för att slita andens band, \\
vi smida medan järnet glöder, \\
med senig arm och kraftig hand.\\ 
\newline
Upp till kamp... \\
\newpage 
\noindent
I sin förgudning avskyvärda, \\
månn' guldets kungar nå'nsin haft \\
ett annat mål är att bli närda \\
av proletärens arbetskraft? \\
Vad hen skapat under nöd och vaka \\
utav tjuvar rånat är, \\
när folket kräva det tillbaka \\
sin egen rätt de blott begär. \\
\newline
Upp till kamp... \\
\newline
Båd' stat och lagar oss förtrycka, \\
vi under skatter digna ner. \\
Den rike inga plikter trycka, \\
den arme ingen rätt man ger. \\
Länge nog som myndlingar vi böjt oss, \\
jämlikheten skall nu bli lag. \\
Med plikterna vi hittills nöjt oss, \\
nu taga vi vår rätt en dag. \\
\newline
Upp till kamp... \\
\newline
Till krigets slaktande vi dragits, \\
vi mejats ned i jämna led. \\
För furstars lögner har vi slagits, \\
nu vill vi skapa evig fred. \\
Om de oss driver, dessa kannibaler, \\
mot våra grannar än en gång, \\
vi skjuter våra generaler \\
och sjunger broderskapets sång. \\
\newpage
\noindent
Upp till kamp... \\
\newline
Arbetare, i stad på landet, \\
en gång skall jorden bliva vår \\
När fast vi knyta brodersbandet, \\
då lättingen ej råda får. \\
Många rovdjur på vårt blod sig mätta \\
men när vi nu till vårt försvar, \\
en dag en gräns för dessa sätta, \\
skall solen stråla lika klar. \\
\newline
Upp till kamp emot kvalen \\
sista striden det är. \\
ty Internationalen \\
åt alla lycka bär. \\
Upp slå ned allt det gamla,\\
tålamodet är slut!\\
För frihet vi oss samla,\\
förgöra våldets krut!
\end{lyrics}
\auth{Originaltext: Eugène Pottier\\
Musik: Pierre Degeyter\\
Svensk text: Henrik Menander}

\nysida{1}{5}
\begin{center}
    \songtitle{$\alpha5$}{Auld lang syne}
%\small\textit{Noter finns i notkapitlet.}
\end{center}
\begin{lyrics}
Should auld acquaintance be forgot, \\
And never brought to mind? \\
Should auld acquaintance be forgot, \\
And days o' lang syne? \\
For auld lang syne my dear, \\
For auld lang syne, \\
We'll take a cup o' kindness yet \\
For auld lang syne.
\vspace{10pt}\\
And there's a hand, my trusty friend, \\
And gie's a hand o' thine \\
We'll take a cup o' kindness yet \\
For auld lang syne. \\
For auld lang syne my dear, \\
For auld lang syne, \\
We'll take a cup o' kindness yet \\
For auld lang syne.
\end{lyrics}

\nysida{1}{}
\begin{center} % Om du ändrar i dessa texter, ändra även motsvarande i /parser/inject/02/
    \large Ordningsregler för gasque enl. fkm* % TODO: Add fkm* logo
\end{center}
\small
1. Gasquer börjar klockan 19. Om annat anges börjar festen ändå klockan 19 eftersom det är då gästerna kommer. \vspace{5pt} \\
2. Regler skall följas. Om de inte följs betraktas det som ett undantag och bekräftar således regeln.
\vspace{5pt} \\
3. På gasque är klädsel starkt rekommenderad. Om ej annat anges är overall att betrakta som klädsel, och b-frack eller dylikt att betrakta som overall. Om gasquen hålls utanför KTH är det osannolikt att denna regel gäller.
\vspace{5pt} \\
4. Festdisciplin skall råda, om ej toastmaster anger annat.
\vspace{5pt} \\
5. Med festdisciplin avses: toastmastern skall gå att uppfatta om en så önskar. Gyckel skall kunna höras av alla, oavsett om de vill höra eller inte.
\vspace{5pt} \\
6. Om det är trångt, vilket det är på alla riktiga gasquer, assistera gärna serveringen på uppenbara sätt så går allting fortare.
\vspace{5pt} \\
7. Om kastvänlig mat serveras, så som köttbullar, citronfromage, pasta e.d., skall detta ses som ett uppriktigt misstag från arrangörernas sida och inte som en inbjudan till matkrig.
\vspace{5pt} \\
8. Har en för avsikt att komma i tid kommer en i tid. Annars kommer en lämpligtvis för sent.
\end{document}