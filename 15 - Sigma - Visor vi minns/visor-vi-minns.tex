\documentclass[a6paper, 10pt, twoside]{article}
%\usepackage[T1]{fontenc}
\usepackage[british]{babel}
\usepackage[utf8]{inputenc}
\usepackage{float, graphicx,amsmath,amsfonts,cite,enumerate,tabularx}
\usepackage[final]{pdfpages}
\usepackage{wrapfig}
\usepackage[margin=0.3in]{geometry}
\usepackage{sidspaltHack}
\usepackage{digital}
\usepackage{ulem}


\setlength{\evensidemargin}{-0.47in}
\setlength{\oddsidemargin}{-0.37in}
\setlength{\textwidth}{215pt}

\pagestyle{empty}

\begin{document}
\nysida{15}{}
\noindent
\chaptertitlenobr{$\Sigma\sigma$}{Visor vi minns}
\small
\vspace{10pt}

\noindent
\small
Du har nått det sista sångkapitlet. Sigma! 

Häri finner du ett axplock av visor från den gångna året som försöker \textit{summera} de upplevelser vi haft den senaste tiden på Fysiksektionen. Det är inte i närheten av alla fantastiska gyckel och nya sånger som dykt upp under läsåret 2021/2022 som finns med här, men en fingervisning om det som lämnat F-teknologernas strupar hittar du på kommande sidor. 

Det senaste året har varit en fantastisk vandring tillbaka in i ljuset efter en lång tids nedstängning för sektionen och dess sångskapande. Nästa år hoppas vi på ett än mer utförligt sådant här kapitel, så se till att sjunga på under året och se till att tipsa nästa års sångboksansvarig när det är dags för nya tillskott till sångboken.

Jag önskar mycket sångglädje och ännu ett läsår fullt av fantasi, festande och skrålande!
\auth{Björn Thiberg \\Fcoms ordförande 2022}
\nysida{15}{22.1}

\begin{center}
\huge{\textit{2021/2022}}
\end{center}

\begin{center}
    \songtitle{$\sigma22.1$}{Öppna PyCharm}
    \mel{Katyuscha} 
\end{center}
\begin{lyrics}
Öppna PyCharm, klona ner ditt repo \\
Börja koda, från början till slut. \\
$\|$: Lägg på stacken, rekursiva anrop, \\
låt dem byggas upp och räknas ut :$\|$
\vspace{5pt}\\
Testa koden, felsök efter felen, \\
kommentera varenda funktions. \\
\textbf{Kör sen koden och använd den för syftet, \\
spela schack, gå lite överkurs. \\
Kör sen koden och använd den för syftet, }
\vspace{5pt}\\
"Markus, kan du prata lite lägre?"
\vspace{5pt}\\
Ladda upp...\\
Add, commit och push\\
\end{lyrics}
\auth{Klas Kristensson \& Martin Siklosi, F-21}

\nysida{15}{22.2}
\begin{center} % TODO: Double-check digital version formatting
    \songtitle{$\sigma22.2$}{Albarevisan}
    \mel{Kalmarevisan} 
    \instruction{Försångare/\textbf{Alla}}
\end{center}
\begin{lyrics}
\small Uti Albares sal\\
ja där håller lilla Mats\\
\textbf{föreläsningen!}
\vspace{4pt}\\
Isoterm...\\
\textbf{...Isobar}\vspace{5pt}\\
Isokor...\\
\textbf{...ideal gas}\vspace{5pt}\\
Adiabat process!\\
\textbf{Adiabat process!}\vspace{5pt} \\
\textbf{För uti Albare sal,\\ ja där håller lilla Mats\\ föreläsningen!}\vspace{5pt}\\
$\|$: Var är heliumet du köpt? \\
\textbf{Ja, det har jag råkat släppt :$\|$ \\i en långsam process.}
\vspace{4pt} \\
\textit{Isoterm...} (etc.) 
\vspace{4pt} \\
$\|$: Vad betyder entropi\\
\textbf{Men, det vet väl inte vi:$\|$ \\är det energi? (NEJ)}
\vspace{4pt} \\
\textit{Isoterm...} (etc.)
\vspace{4pt} \\
$\|$:Temperaturens definition\\
\textbf{ja, den har jag skrivit om :$\|$ \\ för en femte gång!}
\vspace{4pt} \\
\textit{Isoterm...} (etc.)
\end{lyrics}
\auth{Klas Kristensson \& Martin Siklosi, F-21}


\nysida{15}{22.3}

\begin{center}
    \songtitle{$\sigma22.3$}{En fredagspubrunda}
    \mel{När solen lyser klart på himlen blå} 
\end{center}
\begin{lyrics}
När veckan till sist slutet börjat nå, fallera!\\
Vi på vandring genom Bergs och Farkost gå, fallera!\\
Under rus med glada skratt vi krypa fram, fallera!\\
Ja vi dricka med varandra ner på Sam, fallera!
\vspace{5pt}\\
Längs backarna ska höras vårat klamp, klamp, klamp\\
vårat stamp, stamp, stamp, natten lång!\\
Och till solen sen i öster reser sig, sig, sig\\
hörs vårt hej, hej, hej, och våran sång!
\vspace{5pt}\\
På Kemi så kastas pingisbollar runt, fallera!\\
Och på Indek kastar nån en sedelbunt, fallera!\\
Hos Maskin så tömmer nån en öl i hals, fallera!\\
Men på Fysik ser jag faktiskt ingen alls, fallera!
\vspace{5pt}\\
Längs backarna ska höras...
\vspace{5pt}\\
Sist på Data frågar nån om min LinkedIn, fallera!\\
Börjar fatta att jag knappt kan se någonting, fallera!\\
Stad i ljus och sen en kvällens sista rit, fallera!\\
Bort mot Handels skriker vi att de är skit, fallera!
\vspace{5pt}\\
Mot portarna ska höras vårat bank, bank, bank\\
vårat klank, klank, klank, vårat vrål!\\
Och när glasen sen kring bordet höjer sig, sig, sig\\
hörs vårt hej, hej, hej, och våran skål!\\
\end{lyrics}
\auth{Philip Sjögren, F-19 \\Stuggasquen 2021}


\nysida{15}{22.4}
\setlength{\oddsidemargin}{-0.47in}

\begin{center}
    \songtitle{$\sigma22.4$}{Vi är Fnatt}
    \mel{Hallonsaft - Klasse Möllberg}
    \instruction{Berättare/\textbf{Fnattarna}} % TODO: Double-check digital version formatting
\end{center}
\begin{lyrics}
\textbf{La, la, la la la\\
La la la la la la la\\
La, la, la la la\\
La la la la la la la}
\vspace{5pt}\\
Det började på-en mottagning\\
\textbf{Där vi sprang helt vilt omkring}
\vspace{5pt}\\
Var fönstret öppet på er bil\\
\textbf{rakt-in vi hoppade med stil}
\vspace{5pt}\\
De lekte med knivar varje dag\\
\textbf{Och ledare skulle butlern va}
\vspace{5pt}\\
även om-hon blev krossad till damn\\
\textbf{Så lever-Jannie-Jan-kvar i vår famn}
\vspace{5pt}\\
Fnatt, Fnatt, de är Fnatt\\
Många skratt från deras spratt\\
\textbf{Fnatt, Fnatt, vi är Fnatt}\\
\textbf{Många skratt från våra spratt}
\vspace{5pt}\\
När vågformsläran blir för svår\\
\textbf{Vi sliter av oss allt vårt hår}
\vspace{5pt}\\
De får nog aldrigt något mys\\
\textbf{med envariabelanalys}
\vspace{5pt}\\
Laborationer går aldrig bra\\
\textbf{Vi är bevis på Murphys lag}
\vspace{5pt}\\
Bra betyg de nog aldrig se\\
\textbf{Vad-finns det bokstav före E?}
\vspace{5pt}\\
Fnatt, Fnatt, de är Fnatt\\
Många skratt från deras spratt\\
\textbf{Fnatt, Fnatt, vi är Fnatt}\\
\textbf{Många skratt från våra spratt}
\vspace{5pt}\\
När Fnattar går ut på en gasque\\
\textbf{vi dricker allt från punsch till slask}
\vspace{5pt}\\
Från instruktioner till pastej\\
\textbf{Vi äter allt du-kan tänka dig}
\vspace{5pt}\\
För toastmasters är dem en pest\\
\textbf{För fucka ur gör vi som bäst}
\vspace{5pt}\\
Och innan de bärs ut på bår\\
\textbf{Så skriker vi glatt ut gutår}
\vspace{5pt}\\
Fnatt, Fnatt, de är Fnatt\\
Många skratt från deras spratt\\
\textbf{Fnatt, Fnatt, vi är Fnatt}\\
\textbf{Många skratt från våra spratt}
\vspace{5pt}\\
\textbf{Fnatt, Fnatt, vi är Fnatt}\\
\textbf{Många skratt från våra spratt (x6)}\\
\end{lyrics}
\auth{Benjamin Velin, F-21}


\nysida{15}{22.5}

\begin{center}
    \songtitle{$\sigma22.5$}{Bortom Kons förgätmigej}
    \mel{Marseille}
\end{center}
\begin{lyrics}
Jag går i Osquars backe\\
Så frusen sur och trött\\
Då kommer ej emot mig\\
En teknolog jag aldrig mött
\vspace{5pt}\\
Han viskar i mitt öra\\
Jag har nånting för dig\\
Vet du hur flått man har det\\
Bortom kons förgätmigej
\vspace{5pt}\\
Han sa jag ger dig kunskap\\
Han sa jag gör dig smart\\
Om du har sökt till F\\
Och det hade ju jag
\vspace{5pt}\\
Sen drar vi genom kampus\\
Bortom kons förgätmigej\\
Där tavlorna är svarta\\
Och whiteboards är det ej
\vspace{5pt}\\
Där Mats trollar med matte\\
Och Markus spelar schack\\
Och alla block är whitelines\\
Och fysiker har frack
\vspace{5pt}\\
Med an gav Danijela\\
Uppgift ni borde sätt\\
För assen var försiktig\\
Och gav mig inga rätt
\vspace{5pt}\\
\newpage
\noindent
Ingen hjälp till tentan\\
Är följden som jag ser\\
Och människan jag plugga med\\
Den såg jag aldrig mer
\vspace{5pt}\\
Jag har ett mål med livet\\
Det kan jag lova dig\\
En dag ska jag ex:a\\
Bortom kons förgätmigej
\vspace{5pt}\\
Där Mats trollar med matte\\
Och Markus spelar schack\\
Och alla block är whitelines\\
Och fysiker har frack\\
\end{lyrics}
\auth{Martin Siklosi, F-21}

\end{document}
