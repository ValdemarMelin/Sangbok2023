\documentclass[a6paper,12pt]{article}
%\usepackage[T1]{fontenc}
\usepackage[british]{babel}
\usepackage[utf8]{inputenc}
\usepackage{float, graphicx,amsmath,amsfonts,cite,enumerate,tabularx}
\usepackage[final]{pdfpages}
\usepackage[margin=0.3in]{geometry}
\newcommand{\mel}[1]{\small\textbf{\textit{mel. #1 \\}}}


\setlength{\oddsidemargin}{-0.37in}
\setlength{\textwidth}{225pt}

\pagestyle{empty}

\begin{document}
\noindent Bortsupen av: .............................................
\vspace{50pt}
\begin{figure}[!h]
\centering
\includegraphics[width=\textwidth]{sangbok.png}
\end{figure}
\vspace{-20pt}
\begin{center}
\Huge\textbf{Kongl. Fysiks sångbok} \\
\Large anno 2023
\end{center}

\newpage
\setlength{\oddsidemargin}{-0.57in}
\noindent
\Large Vördade F-teknolog... 

\footnotesize \noindent ...eller annan sångglad person!  Du har nu nöjet att hålla i din hand 2023 års upplaga av Kongl. Fysiks sångbok!

I denna orangea pärm finner du de sånger som traditionellt sjungs på Fysiksektionens gasquer och banquetter.
Sångboken innehåller allt från visor till sälla drycker till nidvisor
till skojiga visor till vackra sånger, allt behändigt
sorterat efter lämpligt (eller olämpligt) döpta kapitel.

En del av sångerna har sjungits på KTH eller andra lärosäten sedan urminnes tider.
Vissa har fått sig en justering i för att ta in dem i nutiden.
Andra är helt nyskrivna - I kapitel $\sigma$ finner du ett axplock av sånger som sjungits
och/eller skapats i anslutning till Fysiksektionen under det gångna året. 

Tack till alla som bidragit i år och alla som tidigare arbetat med sångboken, och tack till dig som sjunger! Att få förvalta denna musikskatt är en
ynnest och ett mycket hedersamt uppdrag som jag utfört med glädje.

Mycket sångglädje önskas, och kom ihåg att \textbf{starkt
är vackert!}

\begin{flushright}
\textit{Valdemar Melin, F-18 Flingsalt\\ Fcoms ordförande 2023}
\end{flushright}

\newpage
\includepdf[pages={1-}]{innehall.pdf}

\newpage
\setlength{\oddsidemargin}{-0.57in}
\noindent % Om du ändrar i dessa texter, ändra även motsvarande i /parser/inject/01/
\Large Vettiquette\\
\footnotesize Vett- och etikettvärlden kan vid en första anblick verka 
snårig med otaliga regler som dessutom inte är skrivna i sten och 
därför är olika lite överallt. Men lugn - du kommer långt på vanlig 
hyfs och, ja, vett och det viktigaste tas upp här. Ettiquetten har 
långa anor och delar av den är förlegad, varför den alltid är i 
långsam förändring. Här presenterar jag en version där alla ska kunna 
känna sig inkluderade, som inte begränsar klädkoden utifrån kön och 
som inte förutsätter att ett visst kön behöver hjälp av ett annat 
utan som går bortom könen. Men mycket har levt kvar då det förhöjer 
stämningen och skapar en högtidlig atmosfär. Kanske är de regler du läser 
här lite annorlunda än vad du sett tidigare, men jag har här försökt 
utgå från hur Fysiksektionen och dess banquetter ser ut.\\
\normalsize\textbf{Klädkod}\\
\footnotesize På banquette är det klädkod högtidsdräkt som gäller. 
Det innebär vanligen balklänning eller frack men även militär högtidsuniform 
och högtidlig folkdräkt går bra. Vad gäller frack är det inga konstigheter 
- en frack är en frack. Balklänning är dock ett bredare begrepp men 
gemensamt för alla balklänningar är att de är golvlånga. Resten, så 
som mönster, färg och utformad spelar ingen större roll, så länge det 
är en festfin klänning. Även tvådelat går bra, så länge kjolen är golvlång.\\
Det finns även en mängd tillbehör som kan bäras vid en banquette, 
så som schmeck, handskar, medaljer, och sidenband. Vad gäller handskar b
ärs den korta 
\newpage
\setlength{\oddsidemargin}{-0.37in}
\noindent
varianten till frack och till klänning är det endast kutymt med handskar 
(och då är det den långa varianten som gäller) om klänningen har korta 
ärmar eller inga alls. Handsken ska ej kombineras med ringar utanpå, 
men en elegant klocka eller ett dito armband går bra. Under måltiden 
tas handskarna av men kan annars behållas på.\\
Schmecken är en studentikos mössa som lämpar sig vid högtidliga 
tillfällen. Till schmecken hör de färg-\\
glada spegaterna. Som fysiker sätter en på en orange spegat för 
varje påbörjat år och för varje frånvaroår en svart spegat. Schmecken kan 
bäras under hela banquetten. Vad gäller andra huvudbonader så ska hatt 
ej kombineras med frackklänning men hatt till frack är okej, och då 
är det cylindervarianten som gäller (även om detta är mycket ovanligt). 
Huvudpr-\\
ydnader som exempelvis diadem och fjädrar är vanligare och kan 
behållas på under hela tillställningen.\\
\normalsize\textbf{Etikett}\\
\footnotesize Banquetten börjar ofta med mingel innan det är dags att gå 
till bords. Då får du tillfälle att ta reda på vem din bordspartner är, 
om du inte går tillsammans med någon det vill säga. När banquetten 
ska börja leder du och din bordspartner varandra till bordet. Har någon 
av er en otymplig klänning är det kutymt att hjälpa till genom att dra 
ut stolen för henom.\\
Vid bordet hör det till god ordning att en konverserar med sin bordspartner, 
speciellt i början. När det är dags att skåla så ska du alltid skåla med 
öppen barm
\newpage
\setlength{\oddsidemargin}{-0.57in}
\noindent
till din partner. Det vill säga har du din partner till vänster om dig 
skålar du med höger arm. Du skålar först med din bordspartner, sedan med 
personen på andra sidan om dig och sist med personen mitt emot. När alla 
har druckit går en sedan i motsatt ordning och avslutar alltså skålen 
med sin bordspartner. Efter middagen är det i sedvanlig ordning dags 
för dans! Då leder ni varandra till dansgolvet och dansar, enligt tradition, 
den första dansen (en dans är två låtar) med varandra.\\
\normalsize\textbf{Vett}\\
\footnotesize Som jag sa i början kommer du långt med vanligt vett och 
lite hyfs, det är till exempel trevligt att hälla upp vatten till sin 
bordpartner om denna fått slut, även om det inte är någon regel. Och som 
jag även sa är reglerna inte skrivna i sten, ej heller dessa på 
Fysiksektionen. Det hör inte till ovanligheterna att en ser en fysiker i 
kortare knälång klänning eller i mörk kostym vid en banquette, och ingen 
kollar snett på dig om du gör det. Att lägga dyra pengar på en balklänning 
eller en frack är inte en självklarhet. Inte heller lär någon bli arg om du 
glömmer bort i vilken ordning du ska skåla eller om du har en fråga om 
etiquettereglerna. Om en person ändå skulle bli arg är det hen som har begått 
det största etiquettebrottet, inte du. Har du någon övrig fundering 
rekommenderar jag dig att vända dig till en äldre fysiker eller 
etiquettdrottningen Magdalena Ribbing.\\
\indent Och med det önskar jag dig en fabulös kväll!
\begin{flushright}
\textit{Anna Karlhede, F-14}
\end{flushright}
\end{document}